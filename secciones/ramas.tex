\section{Ramas}

    \subsection{Qu\'e es una rama y para que sirve?}
        Una rama en GIT es una linea independiente de desarrollo dentro de un repositorio. Nos permite trabajar en caracter\'isticas, ideas o cambios sin afectar la rama principal del proyecto. La rama por defecto en GIT se llama \texttt{master} o \texttt{main}, dependiendo de la configuracion inicial.

        Las ramas son especialmente utiles para:
        \begin{itemize}
            \item Experimentar sin romper el proyecto principal.
            \item Desarrollar nuevas caracteristicas de forma aislada.
            \item Organizar el trabajo entre distintos desarrolladores.
        \end{itemize}

        Cuando creamos una rama, esta copia el estado actual del repositorio, y desde ese punto podemos trabajar de forma independiente. Luego, si los cambios funcionan, podemos integrarlos a la rama principal.

    \subsection{Trabajar con ramas}
        Para comenzar a trabajar con ramas, necesitamos saber como crearlas, listarlas, y cambiarnos entre ellas.

        \subsubsection{Crear una nueva rama}
            Para crear una nueva rama usamos el comando:
            \begin{lstlisting}
    $ git branch nombre-de-la-rama
            \end{lstlisting}

            Esto crea la rama pero no nos mueve a ella. Para cambiarnos, usamos:

            \begin{lstlisting}
    $ git switch nombre-de-la-rama
            \end{lstlisting}

            O tambien para crearla y movernos a esa rama en un solo paso agregamos la bandera \texttt{-c}
            \begin{lstlisting}
    $ git switch -c nombre-de-la-rama
            \end{lstlisting}

        \subsubsection{Ver las ramas disponibles}
            Para ver todas las ramas del repositorio local, usamos:
            \begin{lstlisting}
    $ git branch
            \end{lstlisting}

            Esto mostrara un listado de ramas, y marcar\'a con un asterisco la rama en la que estamos actualmente.

        \subsubsection{Cambiar entre ramas}
            Para cambiar a otra rama existente usamos:
            \begin{lstlisting}
    $ git switch nombre-de-la-rama
            \end{lstlisting}
            
        \subsubsection{Eliminar una rama}
            Para eliminar una rama local usamos:
            \begin{lstlisting}
    $ git branch -d nombre-de-la-rama
            \end{lstlisting}

            Si GIT nos dice que la rama no se puede eliminar porque no ha sido fusionada (merged), y aun asi queremos borrarla, podemos usar la opcion \texttt{-D} (mayuscula):
            \begin{lstlisting}
    $ git branch -D nombre-de-la-rama
            \end{lstlisting}