\section{States (Estados en GIT)}

    \subsection{Los 3 estados en GIT}
        GIT maneja internamente tres estados distintos por los que pueden pasar los archivos dentro de un repositorio:
        \begin{itemize}
            \item \textbf{Modificado (modified):} El archivo ha sido cambiado en el sistema de archivos, pero esos cambios a\'un no han sido registrados por GIT para una futura confirmaci\'on.
            \item \textbf{Preparado (staged):} El archivo est\'a listo para ser confirmado. Ha sido agregado al \'area de preparaci\'on, conocida como ``staging area''.
            \item \textbf{No modificado (unmodified):} El archivo est\'a exactamente igual que en la \'ultima versión confirmada. No hay cambios pendientes.
        \end{itemize}
        Estos estados son importantes porque GIT nos permite controlar exactamente qu\'e cambios queremos guardar y cu\'ales no, utilizando comandos que modifican el estado de los archivos.

    \subsection{C\'omo deshacer un archivo modificado?}
        Si realizamos cambios en un archivo pero luego queremos descartarlos y volver a la versi\'on original (la \'ultima registrada), podemos usar el comando:
        \begin{lstlisting}
            $ git restore <ruta al archivo>
        \end{lstlisting}
        Este comando elimina los cambios hechos en el archivo y lo deja exactamente como estaba en el \'ultimo registro conocido por GIT

    \subsection{Cómo a\'~nadimos archivos al área de staging}
        Para marcar un archivo como ``preparado'' (pasarlo al estado \textbf{staged}), usamos el comando:
        \begin{lstlisting}
            $ git add <archivo>
        \end{lstlisting}
        Tambi\'en podemos añadir todos los archivos modificados de una vez con:
        \begin{lstlisting}
            $ git add .
        \end{lstlisting}
        Esto es \'util cuando tenemos varios cambios y queremos registrarlos juntos m\'as adelante.

    \subsection{Cómo puedo sacar uno o varios archivos del área?}
        Si accidentalmente agregamos un archivo al \'area de staging, pero queremos quitarlo de ah\'i (sin perder los cambios hechos), usamos:
        \begin{lstlisting}
            $ git restore --staged <ruta al archivo>
        \end{lstlisting}
        Esto lo devuelve al estado de modificado, permiti\'endonos decidir luego si lo queremos preparar o no.

        También podemos desagregar todos los archivos preparados de una vez:
        \begin{lstlisting}
            $ git restore --staged .
        \end{lstlisting}
        Este comando limpia el \'area de staging, sin afectar el contenido modificado de los archivos.