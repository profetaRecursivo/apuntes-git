\section{Merge}

    \subsection{Fusionar ramas}
        Una de las operaciones mas comunes en GIT es fusionar (merge) ramas. Esto significa tomar los cambios de una rama y aplicarlos en otra. Lo mas habitual es trabajar en una rama secundaria (por ejemplo, \texttt{feature-x}) y luego fusionarla con la rama principal (\texttt{main}).

        Para fusionar ramas, primero debemos cambiarnos a la rama que queremos actualizar. Por ejemplo, si queremos fusionar \texttt{feature-x} en \texttt{main}, primero hacemos:

        \begin{lstlisting}
    $ git switch main
        \end{lstlisting}

        Luego ejecutamos el comando de fusion:

        \begin{lstlisting}
    $ git merge feature-x
        \end{lstlisting}

        GIT intentara aplicar todos los cambios de \texttt{feature-x} sobre \texttt{main}. Si no hay conflictos, la fusion se realiza automaticamente y se genera un nuevo commit de merge.

        Es importante entender que el commit de merge tiene como padres a los dos ultimos commits de las ramas involucradas, dejando un registro claro del punto en que las ramas fueron unificadas.

        En caso de que GIT no pueda fusionar automaticamente, se requerira una resolucion de conflictos, lo cual veremos en la siguiente secci\'on.
