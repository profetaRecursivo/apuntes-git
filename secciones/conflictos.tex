\subsection{Resolver conflictos}
    Para resolver un conflicto, debemos abrir manualmente los archivos que GIT marco como conflictivos y decidir que cambios mantener.

    GIT marca las partes en conflicto con estas se\~nales especiales dentro del archivo:

    \begin{lstlisting}
    <<<<<<< HEAD
    contenido de la rama actual (donde estas parado)
    =======
    contenido de la rama que se esta fusionando
    >>>>>>> feature-x
    \end{lstlisting}

    El contenido entre \texttt{<<<<<<< HEAD} y \texttt{=======} representa los cambios que existen en la rama actual (en la que estas parado). El contenido entre \texttt{=======} y \texttt{>>>>>>> feature-x} muestra los cambios que vienen desde la rama que estas intentando fusionar (en este caso, \texttt{feature-x}).

    Debemos editar el archivo para quedarnos solamente con el contenido deseado (puede ser una mezcla de ambos o solo uno) y eliminar completamente las marcas de conflicto (\texttt{<<<<<<<}, \texttt{=======}, \texttt{>>>>>>>}).

    Adem\'as, si usamos un editor de texto o vemos el conflicto en terminal con el comando:

    \begin{lstlisting}
    $ git diff
    \end{lstlisting}

    GIT nos puede mostrar diferencias de esta forma:

    \begin{lstlisting}
    <<<<<<< HEAD
    - linea que esta en la rama actual
    + linea que esta en la rama que se esta intentando fusionar
    =======
    \end{lstlisting}

    A veces tambi\'en veremos prefijos como:

    \begin{lstlisting}
    - contenido eliminado
    + contenido a\~nadido
    \end{lstlisting}

    O en el caso de conflicto de lineas:

    \begin{lstlisting}
    @@ -3,7 +3,7 @@
     linea 1
    -linea vieja
    +linea nueva
     linea 2
    \end{lstlisting}

    Estos signos indican que hay diferencias linea por linea. Las lineas que comienzan con \texttt{-} estaban en la rama actual, y las que comienzan con \texttt{+} vienen desde la rama que se quiere fusionar.

    Una vez resueltos todos los conflictos editando los archivos, los agregamos al area de staging con:

    \begin{lstlisting}
    $ git add archivo.txt
    \end{lstlisting}

    Finalmente, para completar la fusion, debemos hacer el commit correspondiente:

    \begin{lstlisting}
    $ git commit
    \end{lstlisting}

    Este commit guardara los cambios resultantes de la resolucion de conflicto. A veces GIT abre un editor para escribir un mensaje indicando que hubo conflictos resueltos manualmente.

    Es muy importante verificar que los archivos funcionan correctamente despues de la resolucion para evitar errores l\'ogicos o de ejecuci\'on.