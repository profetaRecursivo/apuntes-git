\section{Eliminar ramas}

    \subsection{Por qu\'e eliminar ramas?}
        En GIT, las ramas permiten trabajar en diferentes funcionalidades o correcciones de manera aislada. Sin embargo, si no se eliminan cuando ya no se necesitan, el repositorio puede volverse dif\'icil de mantener.
        
        Algunas razones para eliminar ramas:
        \begin{itemize}
            \item Reducir el desorden visual al listar ramas.
            \item Evitar confusiones sobre el estado actual del proyecto.
            \item Mantener una estructura limpia en el flujo de trabajo.
        \end{itemize}

    \subsection{C\'omo podar las ramas?}
        Cuando trabajamos con un repositorio remoto, muchas veces se eliminan ramas en el servidor pero siguen apareciendo en nuestro repositorio local. Para limpiar (``podar'') esas referencias obsoletas podemos usar:
        \begin{lstlisting}
$ git remote prune origin
        \end{lstlisting}
        Esto eliminar\'a las referencias locales a ramas remotas que ya no existen en el servidor remoto. Es una buena pr\'actica hacerlo de vez en cuando si trabajamos con muchas ramas.

    \subsection{C\'omo eliminar ramas de mi repositorio local que ya no se usan?}
        Para eliminar una rama local que ya no necesitamos, usamos:
        \begin{lstlisting}
$ git branch -d nombre-rama
        \end{lstlisting}
        Este comando verifica que la rama ya haya sido fusionada con la rama actual. Si no ha sido fusionada y aún así quieres eliminarla, puedes forzar la eliminaci\'on con:
        \begin{lstlisting}
$ git branch -D nombre-rama
        \end{lstlisting}

        Para asegurarte de que ya no la necesitas, puedes revisar primero si fue fusionada:
        \begin{lstlisting}
$ git branch --merged
        \end{lstlisting}
        Este comando muestra las ramas que ya han sido fusionadas a la actual.

        Si deseas listar las ramas que \textbf{no} han sido fusionadas, puedes usar:
        \begin{lstlisting}
$ git branch --no-merged
        \end{lstlisting}