\documentclass[11pt, twoside]{article}
\usepackage{multicol}
\usepackage[left=1in,
            right=1in,
            top=1in,
            bottom=1in,]{geometry}
\usepackage{graphicx}
\usepackage[spanish, mexico]{babel}
\usepackage{fancyhdr}
\usepackage{float}
\usepackage{listings}
\usepackage{hyperref}
\usepackage{color}
\usepackage{xcolor}

\newcommand{\titulo}{Apuntes de GIT}

\lstdefinelanguage{bash}{
  morekeywords={sudo,git,cd,ls,mkdir,rm,cp,mv,echo},
  sensitive=true,
  morecomment=[l]{\#},
  morestring=[b]",
}

\lstset{
  language=bash,
  basicstyle=\ttfamily\small,
  backgroundcolor=\color{gray!10},
  frame=single,
  showstringspaces=false,
  keywordstyle=\color{blue},
  commentstyle=\color{gray},
}

\title{\textbf{\titulo}}
\date{}

\pagestyle{fancy}

\fancypagestyle{firstpage}{
    \fancyhead{}
    \fancyhead[LE, LO]{
        \textit{Curso de introducci\'on a GIT}\\
        SCESI\\
        Jaime Sebastian Chavarria Fuertes
    }
    \fancyhead[R]{\today \\ \titulo}
}

\fancyhf{}
\fancyhead[RO,LE]{\thepage}
\fancyhead[RE,LO]{\lec}

\begin{document}
    \maketitle

    \vspace{-1cm}
    \thispagestyle{firstpage}
    \section{Introducci\'on}
    \subsection{Qu\'e es un sistema de control de versiones?}
        Un sistema de control de versiones es un software que nos permite registrar los cambios que son realizados en un archivo o un conjunto de archivos a lo largo del tiempo, no es \'util por el hecho de la facilidad de que varias personas trabajen en un mismo proyecto al mismo tiempo, tambien nos ayuda a que en caso de un error podamos volver a versiones pasadas funcionales de dicho proyecto.

    \subsection{Fundamentos de GIT}
        GIT se basa en los repositorios, un repositorio no es mas que el lugar donde se alojan las versiones de un proyecto junto con el log de cambios que se han hecho en ese proyects.
        Dichos repositorios pueden ser tanto locales como remotos
            \begin{itemize}
                \item Locales.

                    Los repositorios locales son los que tenemos en nuestra computadora.

                \item Remoto.

                    Los repositorios remotos son los que se ubican en un servidor, los cuales permiten que varias personas vean nuestro proyecto, que realicen cambios en el y que esos cambios sean sincronizados.
            \end{itemize}

    \subsection{Configuraci\'on inicial}
            Antes de comenzar a usar GIT, se debe realizar ciertas configuraciones
            \subsubsection{Correo electronico}
                Para indicarle a GIT nuestro correo electr\'onico debemos ejecutar este comando:
                    \begin{lstlisting}
                        \$ git config --global user.email ``reemplaza esto por tu correo''
                    \end{lstlisting}
                    Por ejemplo:
                    \begin{lstlisting}
                        \$ git config --global user.email ``jschavarria77@gmail.com''
                    \end{lstlisting}
            \subsubsection{Nombre}
                Para indicarle a GIT nuestro nombre debemos ejecutar el siguiente comando:
                    \begin{lstlisting}
                        \$ git config --global user.name ``reemplaza esto por tu nombre completo''
                    \end{lstlisting}
                Por ejemplo:
                    \begin{lstlisting}
                        \$ git config --global user.name ``Jaime Sebastian Chavarria Fuertes''
                    \end{lstlisting}

    \subsection{Configurar el editor de c\'odigo que abre GIT}
        GIT tiene configurado abrir por defecto el editor de texto \texttt{Vim} para poder modificar los archivos cuando encuentra conflictos o cuando usamos el comando \texttt{\$ git commit} sin la bandera y el argumento: \texttt{-m ``descripcion del commit''}.
        En caso de querer cambiar el editor por defecto disponemos del siguiente comando:
            \begin{lstlisting}
                \$ git config --global core.editor ``el comando para abrir tu editor''
            \end{lstlisting}
        En mi caso voy a colocar el editor nvim, entonces el comando ser\'ia:
            \begin{lstlisting}
                \$ git config --global core.editor ``nvim''
            \end{lstlisting}
    
    \subsection{Comprobar la configuraci\'on de GIT}
        Para poder ver la configuraci\'on que tenemos en git podemos usar el siguiente comando:
            \begin{lstlisting}
                \$ git config --list
            \end{lstlisting}
        Que nos producira una salida de este tipo:
            \begin{lstlisting}
                user.email=202301300@est.umss.edu
                user.name=Jaime Sebastian Chavarria Fuertes
                core.repositoryformatversion=0
                core.filemode=true
                core.bare=false
                core.logallrefupdates=true
            \end{lstlisting}
        Podemos jugar con las banderas de \texttt{list}, por ejemplo:
            \begin{itemize}
                \item Para poder mostrar la unicamente la configuraci\'on global:
                    \begin{lstlisting}
                        \$ git config --global --list
                    \end{lstlisting}
                \item Para poder mostrar la configuraci\'on del repositorio local:
                    \begin{lstlisting}
                        \$ git config --local --list
                    \end{lstlisting}
            \end{itemize}
        Y muchas mas que se pueden revisar en \href{https://git-scm.com/docs/git-config}{este enlace de la documentaci\'on de GIT}
    
    \subsection{C\'omo inicializar un nuevo proyecto en GIT?}
        Para crear un nuevo repositorio local podemos utilizar el comando:
            \begin{lstlisting}
                \$ git init <direccion de la carpeta>
            \end{lstlisting}
        Este comando nos creara una carpeta con el nombre asignado en la direcci\'on.

        Pero si en caso de tener ya un proyecto podemos ejecutar el mismo comando dentro de la carpeta ra\'iz de nuestro proyecto pero sin pasarle de argumento nada.
            \begin{lstlisting}
                \$ git init
            \end{lstlisting}
        Y asi ya habremos comenzado a utilizar GIT al fin.

    \section{states}

    \section{Commits}

    \subsection{Qu\'e es un commit?}
        Un \texttt{commit} en GIT representa un punto en el tiempo que guarda el estado de los archivos que fueron preparados (staged). Es una especie de ``fotograf\'ia'' del proyecto en un momento determinado. 
        Cada commit contiene informaci\'on como:
        \begin{itemize}
            \item El autor que realiz\'o el cambio.
            \item La fecha y hora del commit.
            \item Un mensaje descriptivo de los cambios.
            \item Un hash \'unico que lo identifica.
        \end{itemize}
        Gracias a los commits, es posible regresar a versiones anteriores, comparar diferencias entre estados y colaborar con otros sin sobrescribir trabajo.

    \subsection{C\'omo puedo hacer un commit?}
        Una vez que uno o varios archivos han sido agregados al \texttt{staging area}, se puede registrar ese conjunto de cambios usando el siguiente comando:
        \begin{lstlisting}
            \$ git commit -m "mensaje descriptivo del cambio"
        \end{lstlisting}
        La bandera \texttt{-m} sirve para escribir el mensaje directamente desde la l\'inea de comandos.
        
        Por ejemplo:
        \begin{lstlisting}
            \$ git commit -m "agregada la seccion de introduccion"
        \end{lstlisting}

        Si omitimos \texttt{-m}, GIT abrir\'a el editor configurado (por defecto \texttt{Vim}) para que escribamos el mensaje del commit.
    
        Nosotros lo tenemos configurado como \texttt{nvim}

    \subsection{Qu\'e es el HEAD?}
        \texttt{HEAD} es un puntero que indica el \textbf{commit actual} en el que estamos trabajando. Siempre apunta al \'ultimo commit en la rama activa.
        
        Cuando realizamos un nuevo commit, \texttt{HEAD} se mueve al nuevo commit creado. 
        Si cambiamos de rama, \texttt{HEAD} apuntar\'a al commit m\'as reciente de esa nueva rama.

        Tambi\'en podemos ver el contenido de \texttt{HEAD} con:
        \begin{lstlisting}
            \$ cat .git/HEAD
        \end{lstlisting}
        O podemos ver hash del commit exacto al que apunta:
        \begin{lstlisting}
            \$ git rev-parse HEAD
        \end{lstlisting}

        Saber qu\'e es \texttt{HEAD} es importante para entender c\'omo funcionan los cambios de rama, los rebase, y otros comandos avanzados de GIT.

    \section{Ramas}

    \subsection{Qu\'e es una rama y para que sirve?}
        Una rama en GIT es una linea independiente de desarrollo dentro de un repositorio. Nos permite trabajar en caracter\'isticas, ideas o cambios sin afectar la rama principal del proyecto. La rama por defecto en GIT se llama \texttt{master} o \texttt{main}, dependiendo de la configuracion inicial.

        Las ramas son especialmente utiles para:
        \begin{itemize}
            \item Experimentar sin romper el proyecto principal.
            \item Desarrollar nuevas caracteristicas de forma aislada.
            \item Organizar el trabajo entre distintos desarrolladores.
        \end{itemize}

        Cuando creamos una rama, esta copia el estado actual del repositorio, y desde ese punto podemos trabajar de forma independiente. Luego, si los cambios funcionan, podemos integrarlos a la rama principal.

    \subsection{Trabajar con ramas}
        Para comenzar a trabajar con ramas, necesitamos saber como crearlas, listarlas, y cambiarnos entre ellas.

        \subsubsection{Crear una nueva rama}
            Para crear una nueva rama usamos el comando:
            \begin{lstlisting}
                \$ git branch nombre-de-la-rama
            \end{lstlisting}

            Esto crea la rama pero no nos mueve a ella. Para cambiarnos, usamos:

            \begin{lstlisting}
                \$ git switch nombre-de-la-rama
            \end{lstlisting}

            O tambien para crearla y movernos a esa rama en un solo paso agregamos la bandera \texttt{-c}
            \begin{lstlisting}
                \$ git switch -c nombre-de-la-rama
            \end{lstlisting}

        \subsubsection{Ver las ramas disponibles}
            Para ver todas las ramas del repositorio local, usamos:
            \begin{lstlisting}
                \$ git branch
            \end{lstlisting}

            Esto mostrara un listado de ramas, y marcar\'a con un asterisco la rama en la que estamos actualmente.

        \subsubsection{Cambiar entre ramas}
            Para cambiar a otra rama existente usamos:
            \begin{lstlisting}
                \$ git switch nombre-de-la-rama
            \end{lstlisting}
            
        \subsubsection{Eliminar una rama}
            Para eliminar una rama local usamos:
            \begin{lstlisting}
                \$ git branch -d nombre-de-la-rama
            \end{lstlisting}

            Si GIT nos dice que la rama no se puede eliminar porque no ha sido fusionada (merged), y aun asi queremos borrarla, podemos usar la opcion \texttt{-D} (mayuscula):
            \begin{lstlisting}
                \$ git branch -D nombre-de-la-rama
            \end{lstlisting}
    
    \section{Merge}

    \subsection{Fusionar ramas}
        Una de las operaciones mas comunes en GIT es fusionar (merge) ramas. Esto significa tomar los cambios de una rama y aplicarlos en otra. Lo mas habitual es trabajar en una rama secundaria (por ejemplo, \texttt{feature-x}) y luego fusionarla con la rama principal (\texttt{main}).

        Para fusionar ramas, primero debemos cambiarnos a la rama que queremos actualizar. Por ejemplo, si queremos fusionar \texttt{feature-x} en \texttt{main}, primero hacemos:

        \begin{lstlisting}
    $ git switch main
        \end{lstlisting}

        Luego ejecutamos el comando de fusion:

        \begin{lstlisting}
    $ git merge feature-x
        \end{lstlisting}

        GIT intentara aplicar todos los cambios de \texttt{feature-x} sobre \texttt{main}. Si no hay conflictos, la fusion se realiza automaticamente y se genera un nuevo commit de merge.

        Es importante entender que el commit de merge tiene como padres a los dos ultimos commits de las ramas involucradas, dejando un registro claro del punto en que las ramas fueron unificadas.

        En caso de que GIT no pueda fusionar automaticamente, se requerira una resolucion de conflictos, lo cual veremos en la siguiente secci\'on.


    \subsection{Resolver conflictos}
    Para resolver un conflicto, debemos abrir manualmente los archivos que GIT marco como conflictivos y decidir que cambios mantener.

    GIT marca las partes en conflicto con estas se\~nales especiales dentro del archivo:

    \begin{lstlisting}
    <<<<<<< HEAD
    contenido de la rama actual (donde estas parado)
    =======
    contenido de la rama que se esta fusionando
    >>>>>>> feature-x
    \end{lstlisting}

    El contenido entre \texttt{<<<<<<< HEAD} y \texttt{=======} representa los cambios que existen en la rama actual (en la que estas parado). El contenido entre \texttt{=======} y \texttt{>>>>>>> feature-x} muestra los cambios que vienen desde la rama que estas intentando fusionar (en este caso, \texttt{feature-x}).

    Debemos editar el archivo para quedarnos solamente con el contenido deseado (puede ser una mezcla de ambos o solo uno) y eliminar completamente las marcas de conflicto (\texttt{<<<<<<<}, \texttt{=======}, \texttt{>>>>>>>}).

    Adem\'as, si usamos un editor de texto o vemos el conflicto en terminal con el comando:

    \begin{lstlisting}
    $ git diff
    \end{lstlisting}

    GIT nos puede mostrar diferencias de esta forma:

    \begin{lstlisting}
    <<<<<<< HEAD
    - linea que esta en la rama actual
    + linea que esta en la rama que se esta intentando fusionar
    =======
    \end{lstlisting}

    A veces tambi\'en veremos prefijos como:

    \begin{lstlisting}
    - contenido eliminado
    + contenido a\~nadido
    \end{lstlisting}

    O en el caso de conflicto de lineas:

    \begin{lstlisting}
    @@ -3,7 +3,7 @@
     linea 1
    -linea vieja
    +linea nueva
     linea 2
    \end{lstlisting}

    Estos signos indican que hay diferencias linea por linea. Las lineas que comienzan con \texttt{-} estaban en la rama actual, y las que comienzan con \texttt{+} vienen desde la rama que se quiere fusionar.

    Una vez resueltos todos los conflictos editando los archivos, los agregamos al area de staging con:

    \begin{lstlisting}
    $ git add archivo.txt
    \end{lstlisting}

    Finalmente, para completar la fusion, debemos hacer el commit correspondiente:

    \begin{lstlisting}
    $ git commit
    \end{lstlisting}

    Este commit guardara los cambios resultantes de la resolucion de conflicto. A veces GIT abre un editor para escribir un mensaje indicando que hubo conflictos resueltos manualmente.

    Es muy importante verificar que los archivos funcionan correctamente despues de la resolucion para evitar errores l\'ogicos o de ejecuci\'on.

    \section{Eliminar ramas}

    \subsection{Por qu\'e eliminar ramas?}
        En GIT, las ramas permiten trabajar en diferentes funcionalidades o correcciones de manera aislada. Sin embargo, si no se eliminan cuando ya no se necesitan, el repositorio puede volverse dif\'icil de mantener.
        
        Algunas razones para eliminar ramas:
        \begin{itemize}
            \item Reducir el desorden visual al listar ramas.
            \item Evitar confusiones sobre el estado actual del proyecto.
            \item Mantener una estructura limpia en el flujo de trabajo.
        \end{itemize}

    \subsection{C\'omo podar las ramas?}
        Cuando trabajamos con un repositorio remoto, muchas veces se eliminan ramas en el servidor pero siguen apareciendo en nuestro repositorio local. Para limpiar (``podar'') esas referencias obsoletas podemos usar:
        \begin{lstlisting}
$ git remote prune origin
        \end{lstlisting}
        Esto eliminar\'a las referencias locales a ramas remotas que ya no existen en el servidor remoto. Es una buena pr\'actica hacerlo de vez en cuando si trabajamos con muchas ramas.

    \subsection{C\'omo eliminar ramas de mi repositorio local que ya no se usan?}
        Para eliminar una rama local que ya no necesitamos, usamos:
        \begin{lstlisting}
$ git branch -d nombre-rama
        \end{lstlisting}
        Este comando verifica que la rama ya haya sido fusionada con la rama actual. Si no ha sido fusionada y aún así quieres eliminarla, puedes forzar la eliminaci\'on con:
        \begin{lstlisting}
$ git branch -D nombre-rama
        \end{lstlisting}

        Para asegurarte de que ya no la necesitas, puedes revisar primero si fue fusionada:
        \begin{lstlisting}
$ git branch --merged
        \end{lstlisting}
        Este comando muestra las ramas que ya han sido fusionadas a la actual.

        Si deseas listar las ramas que \textbf{no} han sido fusionadas, puedes usar:
        \begin{lstlisting}
$ git branch --no-merged
        \end{lstlisting}

    \section{GitHub}

    \subsection{Git y GitHub: Son lo mismo?}

        Git y GitHub no son lo mismo, aunque est\'an muy relacionados.

        \begin{itemize}
            \item \textbf{Git} es un sistema de control de versiones distribuido, lo que significa que puedes gestionar tu c\'odigo de manera local sin necesidad de conexi\'on a internet. 
            \item \textbf{GitHub} es una plataforma en l\'inea que permite almacenar proyectos Git de manera remota, facilita la colaboraci\'on y proporciona herramientas para gestionar c\'odigo en equipos. Es, en esencia, un servicio de alojamiento basado en Git.
        \end{itemize}

        Git es la herramienta de control de versiones, mientras que GitHub es el servicio donde se aloja y comparte ese c\'odigo.



    \subsection{Repositorios Remotos}

        Un \textbf{repositorio remoto} es una versi\'on del repositorio que se encuentra alojada en un servidor, generalmente en la nube (como en GitHub). Estos repositorios permiten la colaboraci\'on en l\'inea y sirven para tener una copia de seguridad del proyecto.

        Para trabajar con un repositorio remoto, se usa el comando \texttt{git remote}. Este comando se utiliza para vincular un repositorio local con uno remoto, lo que te permitir\'a sincronizar los cambios entre ambos.

        \begin{lstlisting}
    git remote add origin https://github.com/usuario/repositorio.git
        \end{lstlisting}

        Este comando agrega un repositorio remoto llamado \texttt{origin} que apunta a la URL del repositorio en GitHub.



    \subsection{Clonando un Repositorio Remoto Creado Previamente}

        Si ya existe un repositorio remoto en GitHub y deseas obtenerlo en tu computadora, puedes clonarlo utilizando el siguiente comando:

        \begin{lstlisting}
    git clone https://github.com/usuario/repositorio.git
        \end{lstlisting}

        Este comando crea una copia local exacta del repositorio remoto. Con esto, ya puedes comenzar a trabajar en tu copia local, haciendo cambios y luego subi\'endolos al repositorio remoto.



    \subsection{¿C\'omo Enlazar un Repositorio Local con uno Remoto?}

        Si ya tienes un repositorio local y quieres vincularlo a un repositorio remoto en GitHub, debes usar el siguiente comando:

        \begin{lstlisting}
    git remote add origin https://github.com/usuario/repositorio.git
        \end{lstlisting}

        Este comando vincula tu repositorio local con el repositorio remoto en GitHub, de modo que puedas hacer operaciones como \texttt{push} y \texttt{pull}.



    \subsection{Escribiendo en el Repositorio Remoto}

        Para subir tus cambios al repositorio remoto, utilizas el comando \texttt{git push}. Este comando empuja los cambios locales de tu rama actual al repositorio remoto.

        Si tu repositorio local est\'a vinculado correctamente con el remoto, solo tienes que escribir:

        \begin{lstlisting}
    git push origin main
        \end{lstlisting}

        Esto enviar\'a tus cambios de la rama \texttt{main} al repositorio remoto llamado \texttt{origin}.


    \subsection{No me deja hacer \texttt{push}, me dice que el cambio ha sido rechazado}

        Este error generalmente ocurre porque hay cambios en el repositorio remoto que no est\'an presentes en tu repositorio local. Git te impedir\'a hacer un \texttt{push} si tu rama local est\'a detr\'as de la rama remota.

        La soluci\'on m\'as com\'un es hacer un \texttt{git pull} para traer los cambios del repositorio remoto antes de hacer el \texttt{push}. El flujo de trabajo ser\'ia el siguiente:

        \begin{lstlisting}
    git pull origin main
    git push origin main
        \end{lstlisting}

        El comando \texttt{git pull} traer\'a los cambios del repositorio remoto y los fusionar\'a con tus cambios locales. Despu\'es de esto, ya podr\'as hacer el \texttt{push} sin problemas.


    \subsection{Creando una Rama Remota}

        Para crear una nueva rama en tu repositorio local y luego subirla al repositorio remoto, debes seguir estos pasos:

        Primero, crea y cambia a una nueva rama:

        \begin{lstlisting}
    git checkout -b nueva-rama
        \end{lstlisting}

        Luego, empuja esta nueva rama al repositorio remoto con el siguiente comando:

        \begin{lstlisting}
    git push -u origin nueva-rama
        \end{lstlisting}

        El par\'ametro \texttt{-u} establece la rama remota \texttt{origin/nueva-rama} como la rama de seguimiento para tu rama local. Esto significa que las futuras operaciones de \texttt{push} y \texttt{pull} se realizar\'an por defecto en esa rama remota.



    \subsection{Explicaci\'on de los Comandos \texttt{push}, \texttt{pull} y \texttt{fetch}}

        Los comandos \texttt{git push}, \texttt{git pull}, y \texttt{git fetch} son fundamentales para trabajar con repositorios remotos en Git.

        \subsubsection{git push}

            El comando \texttt{git push} se usa para enviar tus cambios locales al repositorio remoto. Si tienes commits en tu rama local que a\'un no han sido subidos, puedes usar este comando para ``empujarlos'' al servidor remoto.

            Ejemplo:
            \begin{lstlisting}
    git push origin main
            \end{lstlisting}

            Este comando sube tus cambios de la rama \texttt{main} al repositorio remoto \texttt{origin}.

    \subsubsection{git pull}

        El comando \texttt{git pull} se utiliza para traer los cambios desde el repositorio remoto y fusionarlos con tu rama local. Este comando es una combinaci\'on de \texttt{git fetch} (que descarga los cambios) y \texttt{git merge} (que los fusiona con tu rama local).

        Ejemplo:
        \begin{lstlisting}
    git pull origin main
        \end{lstlisting}

        Este comando descarga y fusiona los cambios desde la rama \texttt{main} del repositorio remoto \texttt{origin} con tu rama local.

    \subsubsection{git fetch}

        El comando \texttt{git fetch} solo descarga los cambios desde el repositorio remoto sin hacer ninguna fusi\'on con tu rama local. Es \'util cuando solo quieres ver qu\'e cambios han ocurrido en el repositorio remoto sin que se apliquen autom\'aticamente a tu c\'odigo local.

        Ejemplo:
        \begin{lstlisting}
    git fetch origin
        \end{lstlisting}

        Este comando descarga todos los cambios del repositorio remoto sin alterar tu rama local.
\end{document}